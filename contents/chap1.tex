\section{緒言}
\labsec{introduction}
近年,複数の移動ロボットから構成される群ロボットシステムの研究が進められている.
群ロボットを活かした応用先として,環境センシング,モニタリングなどがあり.
その中で,本研究では群ロボットによる協調運搬に着目する.
群ロボットの協調運搬として,把持機構を利用し物体を掴んで運搬する掴む方法とロボットの体のみで物体を押すことで協調運搬する押す方法がある~\cite{swarmBot,push-only2}.
掴む方法は,ロボットに搭載された把持機構で物体を掴むので,押す方法より安定に運搬できるが,物体の認識と把持機構の制御が必要である.
一方,押す方法は倒れやすい不安定な物体が運搬できないという課題がある.
そこで本研究では,対象物体の重心付近まで乗り上げることで把持機構なしで倒れやすい不安定な物体を運搬可能なロボット群を提案する.
そして,製作したロボットを用いて他のロボットに乗り上げる実験を行う.
さらに,ロボットの段数および前進力の制御を変更し,物体の安定化性能の検証実験を行う.
%~~~~~~~~~~~~~~~~~~~~~~~~~~~~~~~~~~~~~~~~~~~~~~~~~~~~~~~~~~~~~~~~~~~~~~~~