\section{結言}
\labsec{conclusion}
本研究では,物体の重心付近まで乗り上げてロボットの体のみで支持できるロボット群を開発した.
そして実機実験の結果により,ロボットが他のロボットに乗り上げられること,移動部が移動するとき不安定な物体を支持できることが確認できた.
また,制御することで,より安定な協調運搬も確認できた.

% 本研究では,環状のパンタグラフ構造を持つ車輪を設計し,
% 遠心力を用いることで従来とは異なる1アクチュエータでの
% 車輪の駆動と車輪径の変化を実現した.
% また,アクチュエータを用いずに車輪径の変化を行うことで,
% 構成要素は剛体のみであるが,
% その組み合わせによって機構的な柔軟性が生まれた.
% さらに,この車輪が段差の踏破および狭窄空間への進入に優れることや,
% 片輪の車輪径のみが遠心力によって変化する場合,
% 車輪の回転数によって移動方向が変化することを確認した.